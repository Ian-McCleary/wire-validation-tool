\section{Features}

\subsection{Read/write from CSV and XLS}
\subsubsection{Description}
The software shall read from and write to .csv and .xlsx documents

\subsubsection{Priority}
High 

\subsubsection{Stimulus and Response}
 A user selects the documents from their hard drive through the file picker. The software will read the documents and write the results to a separate file

\subsubsection{Functional Requirements}
 
 The software shall read data from the selected .xlsx/csv documents when selected
 and write output data to an .xlsx document upon completion.
 
 \subsection{Splice Handling}
\subsubsection{Description}
 The software shall output a separate row in the spreadsheet for each output of a splice
\subsubsection{Priority}
High

\subsubsection{Stimulus and Response}
When a spliced wire is detected in the input, the software shall show each output of the splice in a separate row in the output Excel document.

\subsubsection{Functional Requirements}
 The software shall display the outputs of wire splices in separate rows of the output.
 
  \subsection{Output Columns}
\subsubsection{Description}
The software shall output the following information for each wire:
Start Pin, End Pin, Fuse Rating. Wire Size, Circuit Function.
\subsubsection{Priority}
High

\subsubsection{Stimulus and Response}
The software reads this information from the inputted documents and compiles the information about each wire into an output document  as specified in requirement 3.3

\subsubsection{Functional Requirements}
 The software shall compile the above information into one output document.
 
\subsection{Build to Windows}
\subsubsection{Description}
 The software will build to and run on a Windows PC
\subsubsection{Priority}
High
\subsubsection{Stimulus and Response}
The software is operational on the Windows operating system
\subsubsection{Functional Requirements}
The software shall be installable upon, and run on a Windows P.C.

  \subsection{Graphical User Interface}
\subsubsection{Description}
 The software will display a Graphical User Interface (GUI) that allows users to pick files, modify parameters, and see status of processing. 
\subsubsection{Priority}
Medium.

\subsubsection{Stimulus and Response}
The software responds to user input and displays data through a graphical interface.

\subsubsection{Functional Requirements}
The software shall display a GUI that allows users to pick the input files and output location, specify names of columns to include, and show processing status
 
  \subsection{Minimum Wire CSA Check}
\subsubsection{Description}
 The software will read in, determine, and output the minimum Cross Sectional Area (CSA) of a wire through its entire run in the wire harness.
\subsubsection{Priority}
Low (Phase 2).

\subsubsection{Stimulus and Response}
The Cross Sectional Area of a wire is specified in the input and tracked to the output.

\subsubsection{Functional Requirements}
The system shall find the minimum wire CSA from the start point to the end point of the wire.

  \subsection{Wire Overheating Check}
\subsubsection{Description}
The system shall cross-check the fuse rating of the wire with the wire CSA to determine if it could overheat.

\subsubsection{Priority}
Low (Phase 2).

\subsubsection{Stimulus and Response}
Wire CSA and fuse rating are part of the input document(s). A percentage of the fuse rating threshold is specified. Wires are flagged in the output document with indication that they could be subject to overheat.

\subsubsection{Functional Requirements}
The software shall determine whether wires are subject to overheating determined by a user-supplied fuse rating threshold and minumum wire CSA.




 
 
