\section{Business Requirements}
\subsection{Background}
 PACCAR manufactures customized trucks and large vehicles. All of the vehicles built by PACCAR have custom wiring harnesses installed, which may have thousands of possible configurations. Currently, the company manually validates the designs of these harnesses by hand before manufacturing.

\subsection{Business Opportunity}
PACCAR needs a computerized validation procedure to streamline the process for better manufacturing. The application would be used as a replacement for physical testing of wiring harnesses.


\subsection{Business Objectives and Success Criteria}
The harnesses custom designed for the trucks have a large amount of possible permutations. It is more efficient to test the configurations via a computerized application than by hand. This application would be successful if it can accurately test harness configurations faster than a human-lead validation process, and with 100\% accuracy. The interface and workflow should be accessible to all employees that use it.

\subsection{Customer and Market Needs}
 The application will speed up the validation process of these harnesses, which would decrease the time spent on harness validation and increase both the number of harnesses produced. This would in turn increase the speed and accuracy of the overall manufacturing process. In addition to providing the potential for greater validation accuracy.
 

\subsection{Business Risks}
If the validation process isn’t 100\% accurate, then there is a possibility for manufacturing defects which would not only impact PACCAR’s reputation, but also the customer receiving the truck. The solution proposed needs to be faster than the current method, while maintaining accuracy and being accessible to employees. 

